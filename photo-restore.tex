% Options for packages loaded elsewhere
\PassOptionsToPackage{unicode}{hyperref}
\PassOptionsToPackage{hyphens}{url}
%
\documentclass[
]{article}
\usepackage{lmodern}
\usepackage{amssymb,amsmath}
\usepackage{ifxetex,ifluatex}
\ifnum 0\ifxetex 1\fi\ifluatex 1\fi=0 % if pdftex
  \usepackage[T1]{fontenc}
  \usepackage[utf8]{inputenc}
  \usepackage{textcomp} % provide euro and other symbols
\else % if luatex or xetex
  \usepackage{unicode-math}
  \defaultfontfeatures{Scale=MatchLowercase}
  \defaultfontfeatures[\rmfamily]{Ligatures=TeX,Scale=1}
\fi
% Use upquote if available, for straight quotes in verbatim environments
\IfFileExists{upquote.sty}{\usepackage{upquote}}{}
\IfFileExists{microtype.sty}{% use microtype if available
  \usepackage[]{microtype}
  \UseMicrotypeSet[protrusion]{basicmath} % disable protrusion for tt fonts
}{}
\makeatletter
\@ifundefined{KOMAClassName}{% if non-KOMA class
  \IfFileExists{parskip.sty}{%
    \usepackage{parskip}
  }{% else
    \setlength{\parindent}{0pt}
    \setlength{\parskip}{6pt plus 2pt minus 1pt}}
}{% if KOMA class
  \KOMAoptions{parskip=half}}
\makeatother
\usepackage{xcolor}
\IfFileExists{xurl.sty}{\usepackage{xurl}}{} % add URL line breaks if available
\IfFileExists{bookmark.sty}{\usepackage{bookmark}}{\usepackage{hyperref}}
\hypersetup{
  pdftitle={photo-restore},
  hidelinks,
  pdfcreator={LaTeX via pandoc}}
\urlstyle{same} % disable monospaced font for URLs
\setlength{\emergencystretch}{3em} % prevent overfull lines
\providecommand{\tightlist}{%
  \setlength{\itemsep}{0pt}\setlength{\parskip}{0pt}}
\setcounter{secnumdepth}{-\maxdimen} % remove section numbering

\title{photo-restore}
\author{}
\date{}

\begin{document}
\maketitle

Return to index

\begin{verbatim}
┌─────────┐         ┌─────────┐         ┌─────────┐         ┌─────────┐
└─────────┴─────────┘         └─────────┘         └─────────┴─────────┘
\end{verbatim}

\subsection{Old photo correction}

Although this sites foundation is text, images are essential as well
people expect clear high resolution images but those in your family or
royaalty free collections are often faded and grainy.

Not only are powerful image editors like photoshop readily available but
also very good open source editors like Gimp, Kritta etc.

Even more exciting recently are powerful A.I. algorithm that can take
your grandma's tiny low resolution collection and scale them up to less
accurate but much more usable higher resolution.

Two places to try up-scaling for free

\textbf{www} myHeritage Photo Enhancer

These algorithms are often most suitable to run on an online service,
not your PC.

\subsubsection{Services to explore}

\textbf{www} algorithmia

\textbf{www} converter

\textbf{www} Restoring

\textbf{www} Colab Research Depixelizer

DeOldify

\subsubsection{Notes}

\textbf{www} Colab Research Depixelizer

\begin{enumerate}
\def\labelenumi{\arabic{enumi}.}
\tightlist
\item
  Crop
\item
  Patch
\item
  Levels
\item
  mask/blur, correct bright or dark areas
\end{enumerate}

\begin{verbatim}
  ┏━━━━┓    ┏━━━━┓    ┏━━━━┓    ┏━━━━┓    ┏━━━━┓    ┏━━━━┓  
  ┗━━━━┛    ┗━━━━┛    ┗━━━━┛    ┗━━━━┛    ┗━━━━┛    ┗━━━━┛  
\end{verbatim}

About histograms:

\textless a
href="http://www.marginalsoftware.com/HowtoScan/using\_histograms\_as\_a\_tool3.htm

\begin{itemize}
\item
  Open the combo box list and select a hypothetical source image to be
  scanned (e.g. Low-key - 2).
\item
  Adjust exposure until the maximum number of tones is reached. Usually
  this reached when the specular highlights (the right-most edge of the
  source histogram) nears the right margin.
\item
  Applying a black-white point setting displays any possible
  discontinuities of the tonal scale due to this function.
\end{itemize}

\begin{verbatim}
  ┏━━━━┓    ┏━━━━┓    ┏━━━━┓    ┏━━━━┓    ┏━━━━┓    ┏━━━━┓  
  ┗━━━━┛    ┗━━━━┛    ┗━━━━┛    ┗━━━━┛    ┗━━━━┛    ┗━━━━┛  
\end{verbatim}

\begin{itemize}
\tightlist
\item
  No matter how new or old your slide is, your scanner will scan the
  gloss that comes off a slide.
\end{itemize}

This leaves you with flat, faded colours. Yes, time also fades colour.
But the biggest issue is the scanning process itself -\/- it's not
perfect.

To get rid of this without touching any of the real colours of your
image, do the following...

\begin{itemize}
\tightlist
\item
  GIMP \textgreater{} Colors \textgreater{} Auto \textgreater{} White
  Balance
\item
  Once you got rid of the glare, you can begin to brighten the colours.
  Go...
\item
  GIMP \textgreater{} Colors \textgreater{} Hue and Saturation
\item
  It's very simple, go...
\item
  Colors \textgreater{} Levels
\item
  I know I'm telling you to use "Color Levels" to fix exposure. But in
  my experience, this does a far better job than the actual Exposure
  Levels tool.
\item
  Anyway, here's exactly what to do...
\end{itemize}

Tips On How To Use The Levels Tool

\begin{itemize}
\item
  Here's a few techniques to remember when using the Levels tool...
\item
  Do you see the "Input Levels"?
\item
  Take the middle arrow (the gray one), and move it to the right
\item
  In my scan I moved it to about 0.57-\/- toward the white arrow
\item
  Say your slide is too dark, then move the gray arrow left,toward the
  black arrow
\item
  That's all it takes!
\end{itemize}

\end{document}
