% Options for packages loaded elsewhere
\PassOptionsToPackage{unicode}{hyperref}
\PassOptionsToPackage{hyphens}{url}
%
\documentclass[
]{article}
\usepackage{lmodern}
\usepackage{amssymb,amsmath}
\usepackage{ifxetex,ifluatex}
\ifnum 0\ifxetex 1\fi\ifluatex 1\fi=0 % if pdftex
  \usepackage[T1]{fontenc}
  \usepackage[utf8]{inputenc}
  \usepackage{textcomp} % provide euro and other symbols
\else % if luatex or xetex
  \usepackage{unicode-math}
  \defaultfontfeatures{Scale=MatchLowercase}
  \defaultfontfeatures[\rmfamily]{Ligatures=TeX,Scale=1}
\fi
% Use upquote if available, for straight quotes in verbatim environments
\IfFileExists{upquote.sty}{\usepackage{upquote}}{}
\IfFileExists{microtype.sty}{% use microtype if available
  \usepackage[]{microtype}
  \UseMicrotypeSet[protrusion]{basicmath} % disable protrusion for tt fonts
}{}
\makeatletter
\@ifundefined{KOMAClassName}{% if non-KOMA class
  \IfFileExists{parskip.sty}{%
    \usepackage{parskip}
  }{% else
    \setlength{\parindent}{0pt}
    \setlength{\parskip}{6pt plus 2pt minus 1pt}}
}{% if KOMA class
  \KOMAoptions{parskip=half}}
\makeatother
\usepackage{xcolor}
\IfFileExists{xurl.sty}{\usepackage{xurl}}{} % add URL line breaks if available
\IfFileExists{bookmark.sty}{\usepackage{bookmark}}{\usepackage{hyperref}}
\hypersetup{
  pdftitle={perfect-binding},
  hidelinks,
  pdfcreator={LaTeX via pandoc}}
\urlstyle{same} % disable monospaced font for URLs
\setlength{\emergencystretch}{3em} % prevent overfull lines
\providecommand{\tightlist}{%
  \setlength{\itemsep}{0pt}\setlength{\parskip}{0pt}}
\setcounter{secnumdepth}{-\maxdimen} % remove section numbering

\title{perfect-binding}
\author{}
\date{}

\begin{document}
\maketitle

\section{Identity2}

\begin{verbatim}
 ┌─┐┌┬┐┌─┐┬    ┌─╮ ┬ ┬ ┬┌┬┐ ┬ ┬ ┬ ┌─┐
 └─┐││││ ││    ├┘┐ │ │╲│ ││ │ │╲│ │┌┐
 └─┘┴ ┴└─┘└─┘  ┴─┘ ┴ ┴ ┴ ┴┘ ┴ ┴ ┴ └─┘ 
\end{verbatim}

Return to index

\subsection{Perfect binding}

Text of voice track for youtube video

"Perfect binding" is the binding used in mass market paperback books
where the cut or folded edge of pages are coated with hot melt glue and
the cover is folded over the melted binding to attach it.

Most tutorials are bookbinding artists and enthusiasts who introduce you
to the language and culture of book-binding,I am going to use plain
language and help you see that it is actually simple and inexpensive to
bind as book.

Physical books are not dead, I travel a lot and missed transfers,
quarantine, lost bags etc mean you can easily end up in a strange
country with no charged phone so I have a passport sized book with all
my emergency needs with passport. I create a few heirloom items like
family histories, journals, travel records that I give as gifts and will
still be around in 100 years when no knows what a pdf or ebook is.

Once you can hand make a book you likely are able to send the same files
to a print on demand service on the internet and be a published author.

There are many machines and pre-folded covers available on the market
for small business and home use. These are suitable for the gap between
producing a few small size books like mine in the photo below, and using
a commercial print-on-demand service. I do not do that, I will describe
the small volume DIY process below. It takes a little practice but is a
simple task.

\textbf{img} DIY Passport to Novel size jpg

\subsubsection{DIY tutorial}

I will assume you want a book around large novel size and for
convenience I will assume a home printer capable of A4 and A5 double
sides (manual or auto)

Print the A4 cover on the thickest card-stock your printer can handle,
usually at that's at least 250gm/sqm, or if you have pre-printed card
stock, check that it is not plastic coated or temperature sensitive by
testing a sample of it to 250°C minimum to see it does not bubble or
buckle (tell you how later) I find cutting up paper shopping bags for
expensive stores works well.

Print the A5 pages double-sided in normal 1-2, 3-4, etc. You CAN print
2-up on A4 double-sided and fold them to A5, but I find life is too
short for the extra maths, complication, folding and clamping, unless
your printer cannot handle A5 forget it.

In another place I will talk type-setting, for now just know you will
need to set a narrower left/right margin for front/back of page unless
the book is quite narrow, You will need large margins if you want it
novel size as you will be trimming of a lot of waste. DO check you are
getting front and back text block aligned properly by holding page up to
a window.

The simplest machine is a domestic iron on its back and a sheet of
paper, oven baking paper is best, but any printer paper is OK. The
temperature setting on the iron is not calibrated, so you work around it
by cutting a sliver off the hot-melt glue stick, put them on the paper
and the paper on the iron preheated to maximum and time how long it
takes to melt. If it melts in less than a couple of minutes reduce the
temperature until it does.

This works well, the draw-backs are, You are limited to about A5 paper,
you need about four hands i.e. 2 people for the procedure, its easy to
burn yourself, and the owner of the iron needs to share your enthusiasm.

My current simple machine is a short piece of rectangular steel tube
stabilized in a rudimentary wooden frame that has holes that take
take-away chopsticks to hold th cover in place while I work with two
hands only. The pipe is heated by a hot air blower poked in one end so
it has accurate control from room temp up to 450°C \textbf{img} Simple
binding machine to keep glue hot

Print your A5 pages, check that font and back texts are all aligned,
insert any blank and artwork and photo pages, an extra plain page front
and back can sometimes save you from mistakes as you can cut them away
or glue them down.

Press the block of pages firmly together and measure the thickness.

Print the cover if you wish, or cut your pre-printed card to about A4
when score two grooves near the center of the page per below so the gap
INSIDE the scores is the same as the thickness. Find a narrow but smooth
blade shape tool (I use the back of a table knife) that will compress
and mark the card without cutting and weakening it, the objective is to
score until you can make a sharp straight fold at that point.

\begin{verbatim}
┌──────────────────────┬─┬──────────────────────┐
│                      │ │                      │
│  A4 Paper            │ │                      │
│                      │ │                      │
│                      │ │                      │
│                      │ │                      │
│                      │ │                      │
│                      │ │                      │
│                      │ │                      │
│                      │ │                      │
│                      │ │                      │
│                      │ │                      │
│                      │ │                      │
│                      │ │                      │
│                      │ │                      │
│                      │ │                      │
└──────────────────────┴─┴──────────────────────┘
                       ↑ ↑
                       │ │
            Score two grooves in paper
\end{verbatim}

Fold and unfold until these hinges open and close sharply easily

\begin{verbatim}
    ◄─────────A4 folded about half ────────────►

    ┌───────────────────────────────────────────
┌──►│
│   └───────────────────────────────────────────
│
└──────Measure thickness of printed pages
\end{verbatim}

Now look at a paperback book and observe that the glue fills the space
between the edge of the pages and the cover, and importantly extends
slightly into the face of the first page.

To replicate this, you need to apply glue evenly over either in the
"gutter" we folded in the cover or on the edge of the stack of pages.
Many of the people training on this run a bead of hot glue down the
spine, run a blade down it and evenly spread it right across the spine,
no heater, then push the spine into the cover and then press it until it
is cool.

I also want to do it that way, but I don't have the skill, speed and
constant practice to do it successfully, at least not yet.

I place the folded cover in the simple or simpler heating machine, this
allows me to take as long as I like to get the glue perfectly
distributed, as it will stay hot as long as I need.

Prepare the Pages by tamping the spine edge to align the pages, then use
strong Aligator or similar clips or staples if the book is thin enough,
to firmly hold it together.

If the margin is wider than you want because you did not get the printer
settings correct, or if the page edges are still uneven at the spine you
now get out the only two tools you absolutely need, (1) A strong, rigid,
sharp box cutter or similar knife and (2) A metal or other rigid ruler.

With long smooth unforced strokes, slice the edge of the block off to
create the margin you need.

\begin{verbatim}
┌───────────────────────────────── Clips, Staples, Clamps
│
│     ┌───────────────────┐
│     │                   ├┐
│     │                   ││
│    ┌┼┐ ──────────────   ││
└───►└┼┘                  ││
      │  ──────────────   ││
     ┌┼┐                  ││
     └┼┘ ──────────────   ││
      │                   │┤◄─────Spine to be glued
     ┌┼┐ ──────           ││
     └┼┘                  ││
      │  ─────────────    ││
     ┌┼┐                  ││
     └┼┘ ──────────────   ││
      │                   ││
      └─┬─────────────────┘│
        └──────────────────┘
\end{verbatim}

Preheat the glue warmer, say 230°C

Place the cover spine on the heater,

Run a bead of hot glue down the gutter, you can spread it out with
toothpicks, swizzle sticks or whatever you have to get it even all over,
including 1-2 cm around the corner of the spine. Look at the thickness
of the glue strip in a novel and try for one and a half of that, as our
process is less precise and controlled.

Now if you are paying attention you will realize you don't absolutely
need the glue gun as you can slice a strip off a cold glue stick place
it and melt it with the simple heater, it's just easier and less stress
on the cover.

If the temperature is too low, the glue will be too viscous, just turn
it up. If it is too hot your cover material will warp, scorch, bubble or
de-laminate, throw it away and make a new cover, These temperatures are
just a bit higher than boiling water, it won't catch fire.

Practice bending the page stack one way and the other on the long axis
by just twisting the clamped edge, this slides the spine edge pages back
and forward over one another.

Place the spine edge into the trough of the folded cover and press the
spine into the glue to ensure contact then twist the stack back and
forward while holding down firmly, this will drag some glue up into the
pages slightly and capture any that you did not tamp down to perfect
alignment.

Inspect between the cover and first page ear the spine, it should be
glued at least 1 mm up the page, excess glue will stick them together to
far in, but we added a blank page so its not critical, even if you
messed up completely you slice off the spine again later and rebind
losing only the blank page.

Carefully remove from heater, there may be glue blobs on the ends but
that will be trimmed later. Working quickly place on a board and put
another board on top, align the spine with edges of the board then place
this sandwich under a weight, table leg, pile of books, whatever and run
your finger along the spine to ensure the cover spine did not bulge off
the spine when weighted.

As soon as it cools it is at full strength, lay the book on a cutting
surface, a grid pattern helps you keep it square. Using the knife and
ruler carefully cut back the three exposed edges, take your time
unforced smooth strokes always making sure the blade is perpendicular.

Get the ruler and scoring tool and score a line front and back 2-3 mm
from the spine to encourage the book to fold there rather than straining
the glue strip.

\subsubsection{Temperatures for DIY binding}

\begin{itemize}
\tightlist
\item
  Baking paper to 350°C
\item
  500g Cover Set to 350°C
\end{itemize}

Cold pre-glued cover

\begin{itemize}
\tightlist
\item
  2 min 30 seconds fully melt even without full contact
\end{itemize}

Perspex bending

\begin{itemize}
\tightlist
\item
  Set 220°C (increase until distorts)
\end{itemize}

\subsubsection{Some commercial binding machines}

\begin{itemize}
\tightlist
\item
  Unibind XU138, 238, 338
\item
  Unibinder 120 8M and 8.2
\item
  Fellowes Helios 30 and Helios 60
\item
  Covermate 550, ThermoBind TB500
\item
  ThermoBind TB300, ThermoBind TB240S
\item
  ProBind 1000, ProBind 2000
\item
  ProBind Hardcover Crimper, Coverbind 5000,
\item
  Fellowes 250
\item
  Fellowes 450
\item
  Bind-it Perfect Binder II,
\item
  Bind-it Covermate 600 and 700 series
\end{itemize}

\end{document}
