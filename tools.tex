% Options for packages loaded elsewhere
\PassOptionsToPackage{unicode}{hyperref}
\PassOptionsToPackage{hyphens}{url}
%
\documentclass[
]{article}
\usepackage{lmodern}
\usepackage{amssymb,amsmath}
\usepackage{ifxetex,ifluatex}
\ifnum 0\ifxetex 1\fi\ifluatex 1\fi=0 % if pdftex
  \usepackage[T1]{fontenc}
  \usepackage[utf8]{inputenc}
  \usepackage{textcomp} % provide euro and other symbols
\else % if luatex or xetex
  \usepackage{unicode-math}
  \defaultfontfeatures{Scale=MatchLowercase}
  \defaultfontfeatures[\rmfamily]{Ligatures=TeX,Scale=1}
\fi
% Use upquote if available, for straight quotes in verbatim environments
\IfFileExists{upquote.sty}{\usepackage{upquote}}{}
\IfFileExists{microtype.sty}{% use microtype if available
  \usepackage[]{microtype}
  \UseMicrotypeSet[protrusion]{basicmath} % disable protrusion for tt fonts
}{}
\makeatletter
\@ifundefined{KOMAClassName}{% if non-KOMA class
  \IfFileExists{parskip.sty}{%
    \usepackage{parskip}
  }{% else
    \setlength{\parindent}{0pt}
    \setlength{\parskip}{6pt plus 2pt minus 1pt}}
}{% if KOMA class
  \KOMAoptions{parskip=half}}
\makeatother
\usepackage{xcolor}
\IfFileExists{xurl.sty}{\usepackage{xurl}}{} % add URL line breaks if available
\IfFileExists{bookmark.sty}{\usepackage{bookmark}}{\usepackage{hyperref}}
\hypersetup{
  pdftitle={tools},
  hidelinks,
  pdfcreator={LaTeX via pandoc}}
\urlstyle{same} % disable monospaced font for URLs
\setlength{\emergencystretch}{3em} % prevent overfull lines
\providecommand{\tightlist}{%
  \setlength{\itemsep}{0pt}\setlength{\parskip}{0pt}}
\setcounter{secnumdepth}{-\maxdimen} % remove section numbering

\title{tools}
\author{}
\date{}

\begin{document}
\maketitle

\section{Identity2}

\begin{verbatim}
┬┌┬┐┌─┐┬ ┬┌┬┐┬┌┬┐┬ ┬┌─┐  ┬  ┌─┐┬ ┬┌─┐ ┌─┐┌─┐┌─╮┌┬┐ ┌┬┐┌─┐┬ ┬┌┬┐
│ ││├┤ │╲│ │ │ │ └┬┘┌─┘  │  │ ││╲││┌┐│├┤ │ │├( │││  │ ├┤  ╳  │
┴ ┴┘└─┘┴ ┴ ┴ ┴ ┴  ┴ └──  └─┘└─┘┴ ┴└─┘ ┴  └─┘┴ ┴┴ ┴  ┴ └─┘┴ ┴ ┴
\end{verbatim}

\subsection{Writing}

\subsubsection{Tools}

You can write with anything you like as long as it understands unicode.

It is best to use an editor that allows you to write plain text with as
little clutter and distraction as possible, there are hundreds to choose
from. I am currently using Geany as I am using gemtext markup and it
shows formatted view in a sidebar. \textbf{www} Geany. Win Linux Mac

To open the generated files from this site other than .gmi gemtext files
and re-edit them you can use: \textbf{www} {[}tex,pdf{]} Online Overleaf
LaTeX editor

\textbf{www} {[}tex,pdf{]} Install Win/Mac/Linux Texstudio LaTeX editor

\textbf{www} {[}epub,mobi,etc.{]} Install Win/Mac/Linux Sigil eBook
editor

\textbf{www} {[}jpg,png,etc.{]} Install Win/Mac/Linux Image editor

A particular issue with writing in unicode in gemtext with simple tools
is that is easy to enter unicode character that you can not find to
remove but creates hassles down stream, find them like this:

\begin{verbatim}
   find . -type f -exec grep -H '┌┬┐┌─┐┬ ┬' {} \;
\end{verbatim}

If you have no 'nix (That's MacOS, Windows subsystem for 'nix, Linux and
\& BSD) terminal for this or other commands there are many terminal
choices online, including: \textbf{www} Free/no registration terminal if
u have no 'nix

\textbf{www} another Free/no registration terminal

The tools above are what I use, partly because they are all free and
supported by a community. Frankly it does not matter much what you use
when you are aiming for smol and simple systems.

A good text graphics grapics app would be handy but I have never found
one. The good ones use only ascii characters which is too limiting and
ugly and the unicode ones are immature.

\begin{verbatim}
  ┏━━━━┓    ┏━━━━┓    ┏━━━━┓    ┏━━━━┓    ┏━━━━┓    ┏━━━━┓  
  ┗━━━━┛    ┗━━━━┛    ┗━━━━┛    ┗━━━━┛    ┗━━━━┛    ┗━━━━┛  
\end{verbatim}

\end{document}
