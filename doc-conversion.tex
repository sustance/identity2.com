% Options for packages loaded elsewhere
\PassOptionsToPackage{unicode}{hyperref}
\PassOptionsToPackage{hyphens}{url}
%
\documentclass[
]{article}
\usepackage{lmodern}
\usepackage{amssymb,amsmath}
\usepackage{ifxetex,ifluatex}
\ifnum 0\ifxetex 1\fi\ifluatex 1\fi=0 % if pdftex
  \usepackage[T1]{fontenc}
  \usepackage[utf8]{inputenc}
  \usepackage{textcomp} % provide euro and other symbols
\else % if luatex or xetex
  \usepackage{unicode-math}
  \defaultfontfeatures{Scale=MatchLowercase}
  \defaultfontfeatures[\rmfamily]{Ligatures=TeX,Scale=1}
\fi
% Use upquote if available, for straight quotes in verbatim environments
\IfFileExists{upquote.sty}{\usepackage{upquote}}{}
\IfFileExists{microtype.sty}{% use microtype if available
  \usepackage[]{microtype}
  \UseMicrotypeSet[protrusion]{basicmath} % disable protrusion for tt fonts
}{}
\makeatletter
\@ifundefined{KOMAClassName}{% if non-KOMA class
  \IfFileExists{parskip.sty}{%
    \usepackage{parskip}
  }{% else
    \setlength{\parindent}{0pt}
    \setlength{\parskip}{6pt plus 2pt minus 1pt}}
}{% if KOMA class
  \KOMAoptions{parskip=half}}
\makeatother
\usepackage{xcolor}
\IfFileExists{xurl.sty}{\usepackage{xurl}}{} % add URL line breaks if available
\IfFileExists{bookmark.sty}{\usepackage{bookmark}}{\usepackage{hyperref}}
\hypersetup{
  pdftitle={doc-conversion},
  hidelinks,
  pdfcreator={LaTeX via pandoc}}
\urlstyle{same} % disable monospaced font for URLs
\setlength{\emergencystretch}{3em} % prevent overfull lines
\providecommand{\tightlist}{%
  \setlength{\itemsep}{0pt}\setlength{\parskip}{0pt}}
\setcounter{secnumdepth}{-\maxdimen} % remove section numbering

\title{doc-conversion}
\author{}
\date{}

\begin{document}
\maketitle

Return to index

\subsection{Eclectic notes on Doc Conversion}

\subsubsection{LaTeX to html and epub3}

texstudio run htlatex to get html file

Use script to strip out the weird extra empty images and get clean htm

Import to htm to calibre

\begin{itemize}
\tightlist
\item
  Change cover image
\item
  Make distinct name
\item
  meta: add author sort
\item
  look: text: justify, smarten punctuation,
\item
  page: ipad3
\end{itemize}

\textbackslash d is equivalent to {[}0-9{]} \textbackslash w is
equivalent to {[}a-zA-Z0-9\_{]} \textbackslash s is equivalent to any
whitespace

``?'' matches 0 or 1 of the preceding element, ``*'' matches 0 or more
of the preceding element and ``+'' matches 1 or more of the preceding
element

\begin{verbatim}
remove this eroneous output from html

\end{verbatim}

\textless img{[}\textbackslash s{]}+src="ElecticTales{[}0-9{]}+x.png"{[}\textbackslash s{]}+alt="PICT"{[}\textbackslash s{]}?\textgreater{}
\textless img{[}\textbackslash s{]}+src="ElecticTales{[}0-9{]}+x.png"{[}\textbackslash s{]}+alt="PICT"{[}\textbackslash s{]}+class="calibre1"{[}\textbackslash s{]}?/\textgreater{}
\textless img{[}\textbackslash s{]}+src="taleChineseEmbers{[}0-9{]}+x.png"{[}\textbackslash s{]}+alt="PICT"{[}\textbackslash s{]}+class="calibre1"{[}\textbackslash s{]}?/\textgreater{}

\begin{verbatim}
images, use eps or

<a href="https://tug.org/tex4ht/">  __www__ Tug</a>

\end{verbatim}

\subsection{command output comment}

htlatex abc abc.html HTML, bitmap math xhlatex abc abc.html XHTML,
bitmap math mzlatex abc abc.xml XHTML, MathML math oolatex abc abc.sxw
OpenOffice XML (uses MathML math) dbmlatex abc abc.xml DocBook, MathML
math

\begin{verbatim}
``
htlatex tale.tex "xhtml,html5,charset=utf-8" " -cunihtf -utf8"
\end{verbatim}

\textless a href="https://www.lode.de/blog/converting-latex-to-html/

If you are experiencing problems after switching a XeLaTeX project to
pdfLaTeX in the project settings, an adaption of the LaTeX code is
necessary. As we do not want to make the original XeLaTeX code unusable,
we need to add conditional statements. For this, you need to include the
ifxetex package:

\begin{verbatim}
\usepackage{ifxetex}
\end{verbatim}

Then, simply surround XeLaTeX-specific code (or simply code that
produces an error) with a ``\textbackslash ifxetex
\ldots\textbackslash else \ldots\textbackslash fi'' construction. Having
this compatibility allows you to generate PDF files with XeLaTeX, and
also produce HTML documents with pdfLaTeX when you switch compiler
settings.

For example, when generating an HTML file, you cannot include PDF files
or vector graphics. Instead, you have to rely on JPG and PNG image
files. Another application would be if you want to minimize the size of
an existing image file for an e-book. A code might look like this:

\begin{verbatim}
\begin{figure}[H]\centering 
\ifxetex 
\adjustbox{max width=.95\columnwidth, max height=.4\textheight}{ 
\input{images/philosophy-hierarchy} 
} 
\else 
\includegraphics{images/philosophy-hierarchy.png} 
\fi 
\label{c1_ontology-epistemology:fig} 
\end{figure}
\end{verbatim}

\end{document}
