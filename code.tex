% Options for packages loaded elsewhere
\PassOptionsToPackage{unicode}{hyperref}
\PassOptionsToPackage{hyphens}{url}
%
\documentclass[
]{article}
\usepackage{lmodern}
\usepackage{amssymb,amsmath}
\usepackage{ifxetex,ifluatex}
\ifnum 0\ifxetex 1\fi\ifluatex 1\fi=0 % if pdftex
  \usepackage[T1]{fontenc}
  \usepackage[utf8]{inputenc}
  \usepackage{textcomp} % provide euro and other symbols
\else % if luatex or xetex
  \usepackage{unicode-math}
  \defaultfontfeatures{Scale=MatchLowercase}
  \defaultfontfeatures[\rmfamily]{Ligatures=TeX,Scale=1}
\fi
% Use upquote if available, for straight quotes in verbatim environments
\IfFileExists{upquote.sty}{\usepackage{upquote}}{}
\IfFileExists{microtype.sty}{% use microtype if available
  \usepackage[]{microtype}
  \UseMicrotypeSet[protrusion]{basicmath} % disable protrusion for tt fonts
}{}
\makeatletter
\@ifundefined{KOMAClassName}{% if non-KOMA class
  \IfFileExists{parskip.sty}{%
    \usepackage{parskip}
  }{% else
    \setlength{\parindent}{0pt}
    \setlength{\parskip}{6pt plus 2pt minus 1pt}}
}{% if KOMA class
  \KOMAoptions{parskip=half}}
\makeatother
\usepackage{xcolor}
\IfFileExists{xurl.sty}{\usepackage{xurl}}{} % add URL line breaks if available
\IfFileExists{bookmark.sty}{\usepackage{bookmark}}{\usepackage{hyperref}}
\hypersetup{
  pdftitle={code},
  hidelinks,
  pdfcreator={LaTeX via pandoc}}
\urlstyle{same} % disable monospaced font for URLs
\setlength{\emergencystretch}{3em} % prevent overfull lines
\providecommand{\tightlist}{%
  \setlength{\itemsep}{0pt}\setlength{\parskip}{0pt}}
\setcounter{secnumdepth}{-\maxdimen} % remove section numbering

\title{code}
\author{}
\date{}

\begin{document}
\maketitle

\section{Identity2}

\begin{verbatim}
┬┌┬┐┌─┐┬ ┬┌┬┐┬┌┬┐┬ ┬┌─┐  ┬  ┌─┐┬ ┬┌─┐ ┌─┐┌─┐┌─╮┌┬┐ ┌┬┐┌─┐┬ ┬┌┬┐
│ ││├┤ │╲│ │ │ │ └┬┘┌─┘  │  │ ││╲││┌┐ ├┤ │ │├( │││  │ ├┤  ╳  │
┴ ┴┘└─┘┴ ┴ ┴ ┴ ┴  ┴ └──  └─┘└─┘┴ ┴└─┘ ┴  └─┘┴ ┴┴ ┴  ┴ └─┘┴ ┴ ┴
\end{verbatim}

\subsection{Code}

\begin{quote}
This site is a working demonstration of taking a standard gemini capsule
and auto generating and distributing derived MD, Web, LaTeX, PDF, ePub,
etc.
\end{quote}

If you can improve on my ugly Python and the HTML template please share
your ideas, and although the Eisvogel works magically I am sure
something simple and elegant can devised for this specific task.

I am interested in futhering and promoting gemtext encoding /gemini as
an integral part for the complete heirachy of longtext publishing.

This site is processed thru Gemtext to Markdown, Markdown to Html,
Markdown to LaTeX, Latex to PDF, PDF to print. If that all works ok
finalstep of Htm to eBooks should be trivial.

Gemtext encoding is identical to COMMONMARK standard markdown except for
the URL encoding, so the following converts it to commonmark.

\begin{verbatim}
[Python3]
modif = ''
with open(fi, 'r') as i_file:
   while True:
       line = i_file.readline()
       if not line:
           break
       if">"<a href="" in line:</a>

           line">= re.sub('<a href="', '<a href="', line)</a>

           line = re.sub('.gmi  +', '.html">', line)
           line = re.sub('  +', '">', line)
           line = re.sub('\n', '</a>', line)
           line = re.sub('gemini://', 'https://', line)
           modif = f'{modif} {line}\n\n'
       else:
           modif = f'{modif} {line}'              
mdfile.write(modif)
mdfile.close()
\end{verbatim}

With the execption that the html encoded url is ignored by pandoc in the
following

\begin{verbatim}
[Python3]
md = re.sub('.gmi', '.md', fi)
texdir = re.sub('gemini', 'gemini-tex', tex)
title = re.sub('.gmi', '', file)
os.system(
   f'pandoc --standalone --from=commonmark --metadata title="{title}" '
   f'--template tmplt.html {md} -o {htm}'
)
\end{verbatim}

With tmplt.html being something like the following

\begin{verbatim}
[html]
<!doctype html>
<html lang="en">
  <head>
    <meta charset="utf-8">
    <meta name="date" content='$date-meta$'>
    <title>Identity2 $title$</title>
<style>
body{margin-left: 4em; background-color: #ffffff;
 background-image: url("i/Noisy-Grid.jpg");background-repeat;}
              
h1  {color: #003c32; margin-top: -0.3em; margin-left: -2em;
     padding-left: 5em; padding-bottom: 0.5em; 
     padding-top: 0.6em; padding-left: -1em; 
     background-color: #ddd;}

h2  {color: #003c32; margin-top: .3em; padding-bottom: 0.2em;
     padding-top: .5em; padding-left: -1em; 
     background-color: rgba(255,255,255,0.33);}
     
h3  {color: #aad7dc;}

ul  {color: #804000;}

li  {color: #804000;}

blockquote {
    font: 14px/22px normal helvetica, sans-serif; margin-top: 10px;
    margin-bottom: 10px; margin-left: 30px; padding-left: 15px;
    border-left: 6px solid #aad7dc;}

p   {padding-left: 1em;padding-top: .6em;}

.language-ascii-art2 {color: #003c32; display: inline-block; 
    text-shadow: 0 0 5px rgba(100,100,100,0.5);
    background-color: rgba(255,255,255,0.33);}}
</style>
  </head>
  <body>
$body$
  </body>
</html>
\end{verbatim}

I place this html on a standard web server, by automated script of
course.

Similarly:

\begin{verbatim}
[Python3]
os.system(
   f'pandoc -N --pdf-engine=xelatex  --template '
   f'eisvogel -f commonmark --toc {md} -o {pdf}')
\end{verbatim}

Pandoc using XeTeX (as its unicode) and the eisvogel template generates
a standalone PDF of every page using this template.

It will be trivial to concantonate all the pages into a book in the same
script iterator, you need to be sure headings levels are consistent so
that the pdf's generated table of contents makes sense.

Since this is an instruction site the properly formatted and paginated
pdf allows users to print instructions cleanly.

This demonstration can inform creators on the effect of site design on
repurposed documents and gemini server builders on possible feaatures
that can be incorporated as server features.

\begin{verbatim}
  ┏━━━━┓    ┏━━━━┓    ┏━━━━┓    ┏━━━━┓    ┏━━━━┓    ┏━━━━┓  
  ┗━━━━┛    ┗━━━━┛    ┗━━━━┛    ┗━━━━┛    ┗━━━━┛    ┗━━━━┛  
\end{verbatim}

\end{document}
