% Options for packages loaded elsewhere
\PassOptionsToPackage{unicode}{hyperref}
\PassOptionsToPackage{hyphens}{url}
%
\documentclass[
]{article}
\usepackage{lmodern}
\usepackage{amssymb,amsmath}
\usepackage{ifxetex,ifluatex}
\ifnum 0\ifxetex 1\fi\ifluatex 1\fi=0 % if pdftex
  \usepackage[T1]{fontenc}
  \usepackage[utf8]{inputenc}
  \usepackage{textcomp} % provide euro and other symbols
\else % if luatex or xetex
  \usepackage{unicode-math}
  \defaultfontfeatures{Scale=MatchLowercase}
  \defaultfontfeatures[\rmfamily]{Ligatures=TeX,Scale=1}
\fi
% Use upquote if available, for straight quotes in verbatim environments
\IfFileExists{upquote.sty}{\usepackage{upquote}}{}
\IfFileExists{microtype.sty}{% use microtype if available
  \usepackage[]{microtype}
  \UseMicrotypeSet[protrusion]{basicmath} % disable protrusion for tt fonts
}{}
\makeatletter
\@ifundefined{KOMAClassName}{% if non-KOMA class
  \IfFileExists{parskip.sty}{%
    \usepackage{parskip}
  }{% else
    \setlength{\parindent}{0pt}
    \setlength{\parskip}{6pt plus 2pt minus 1pt}}
}{% if KOMA class
  \KOMAoptions{parskip=half}}
\makeatother
\usepackage{xcolor}
\IfFileExists{xurl.sty}{\usepackage{xurl}}{} % add URL line breaks if available
\IfFileExists{bookmark.sty}{\usepackage{bookmark}}{\usepackage{hyperref}}
\hypersetup{
  pdftitle={render-docs},
  hidelinks,
  pdfcreator={LaTeX via pandoc}}
\urlstyle{same} % disable monospaced font for URLs
\setlength{\emergencystretch}{3em} % prevent overfull lines
\providecommand{\tightlist}{%
  \setlength{\itemsep}{0pt}\setlength{\parskip}{0pt}}
\setcounter{secnumdepth}{-\maxdimen} % remove section numbering

\title{render-docs}
\author{}
\date{}

\begin{document}
\maketitle

\section{Identity2}

\begin{verbatim}
┬┌┬┐┌─┐┬ ┬┌┬┐┬┌┬┐┬ ┬┌─┐  ┬  ┌─┐┬ ┬┌─┐ ┌─┐┌─┐┌─╮┌┬┐ ┌┬┐┌─┐┬ ┬┌┬┐
│ ││├┤ │╲│ │ │ │ └┬┘┌─┘  │  │ ││╲││┌┐|├┤ │ │├( │││  │ ├┤  ╳  │
┴ ┴┘└─┘┴ ┴ ┴ ┴ ┴  ┴ └──  └─┘└─┘┴ ┴└─┘ ┴  └─┘┴ ┴┴ ┴  ┴ └─┘┴ ┴ ┴
\end{verbatim}

\subsection{General rendering notes}

The specific of how this document is produced are in the "code" section

\subsubsection{Rendering Ascii/Unicode Art}

\textbf{www} Convert ascii to image

\textbf{www} App, aafigure

\subsubsection{eBooks - ePub}

The basic structure of a ePub book is as follows, I have created eBook,
mobi etc entirely from python scripts and it is possible and relatively
easy. In practice however I found it easier to roughly assemble the book
by script and through the editing stages, however instead of programming
for all the edge cases just open it in \textbf{Sigil} to finalize, test,
audit and final render.

ePub2 hello world

\begin{verbatim}
.
 ├── mimetype                (file) THIS ONE MUST BE UNCOMPRESSED
 ├── META-INF
 │   └── container.xml       (file)
 └── OEBPS
     ├── book.ncx             (file)
     ├── book.opf             (file)
     └── chapter01.xhtml      (file)
\end{verbatim}

The file container.xml serves as a "bootstrap" for the book.

The .opf file contains at a minimum, metadata of the book, the list of
all files (except for mimetype, container.xml, and the .opf) that
compose the book, with their full path, IDs and media-types, and,
lastly, the "linear" list of topics in which the book should be read.

The content of the dc:identifier element of the .opf file has to match
the property of the meta element with name dtb:uid. Otherwise,
\textbf{www} Epub-validate

does not validate the document.

Testing suite: Kindle Previewer 2, Adobe Digital Editions, iBooks, then
on-device with Kindle Fire, Paperwhite, iPad and Android.

Font: serif sans-serif cursive fantasy monospace

\begin{verbatim}
  ┏━━━━┓    ┏━━━━┓    ┏━━━━┓    ┏━━━━┓    ┏━━━━┓    ┏━━━━┓  
  ┗━━━━┛    ┗━━━━┛    ┗━━━━┛    ┗━━━━┛    ┗━━━━┛    ┗━━━━┛  
\end{verbatim}

\end{document}
